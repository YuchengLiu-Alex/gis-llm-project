\documentclass{article}
\usepackage[a4paper, margin=0.75in]{geometry}
\usepackage{graphicx}
\usepackage{titlesec}
\usepackage{amsmath}
\usepackage{graphicx}
\usepackage{float}
\usepackage{booktabs}
\usepackage{array}  
\usepackage{rotating}
\usepackage{pifont} 
\usepackage{subcaption}
\usepackage{setspace}
\usepackage{hyperref}
\usepackage{natbib}  % 引用 BibTeX 需要 natbib 包
\usepackage{multirow} 
\usepackage{titlesec}
\usepackage{array} % 控制列宽
\usepackage{adjustbox} 
\usepackage{ragged2e} % 支持两端对齐
\bibliographystyle{chicago}  % 使用 Chicago (Author-Date) 引用格式

\begin{document}

\begin{titlepage}
    \centering
    \vspace*{2cm}
    {\Huge \textbf{Midterm Progress Report of ChatGIS: GIS-Powered Natural Language Query System}} \\
    \vspace{1.5cm}
    \textbf{Group Name:} ChatDB 70 \\
    \textbf{Course:} DSCI 551 - Foundations of Data Science \\
    \textbf{Instructor:} Dr. Wensheng Wu \\
    \textbf{Institution:} Viterbi School of Engineering, University of Southern California \\
    \textbf{Date:} \today \\
    \vfill
    \textbf{Team Members:} \\
    - [Yucheng Liu] - [Project Designing, Data Collection, Document Writing] \\
    

    \vfill
    \textbf{Team Members Background:} \\
    \begin{justify}

        \textbf{Yucheng Liu} is a Master’s student in Spatial Data Science at USC with a strong background in computer science, geospatial data science, and deep learning. He specializes in knowledge distillation, cloud-based application development, and geospatial analytics. Proficient in Python, SQL, NoSQL, and GIS, he has experience with AWS, TensorFlow,  and ArcGIS Pro. His research focuses on knowledge distillation and multimodal learning in the physiological signals, with publications in ACM MM and IJCAI.
        
    \end{justify}

    \vfill
    \begin{justify}
        \textbf{Project Abstract} - This proposal presents a \textbf{GIS-powered natural language query system} called ChatGIS that integrates \textbf{PostGIS} and \textbf{LLMs} to allow users to retrieve geospatial data using natural language queries. The system translates user queries into \textbf{optimized SQL statements} that efficiently interact with a GIS database, enabling users to ask spatial questions such as \textit{``Where are the nearest electric vehicle charging stations?''} or \textit{``Find all Chinese restaurants within 5 km of my location.''}. ChatGIS levels up the accessibility and usability of geospatial data by leveraging \textbf{LLM-based query processing, spatial indexing, and GIS visualization tools}.
        \vfill
        \textbf{Keywords:} GIS, LLM, PostGIS, Natural Language Queries, Spatial Database, Geospatial Search, Route Optimization, OpenStreetMap, API Development, Spatial Indexing, Machine Learning, Location-Based Services
    \end{justify}

\end{titlepage}

\section{Implementation}


\section{Planned Implementation}


\section{Project Status}


\section{Challenges Faced}


\section{Timeline}

% 参考文献部分
\bibliography{references}  % 引用 BibTeX 文件 "references.bib"

\end{document}