\documentclass{article}
\usepackage[a4paper, margin=1in]{geometry}
\usepackage{graphicx}
\usepackage{titlesec}
\usepackage{amsmath}
\usepackage{graphicx}
\usepackage{float}
\usepackage{booktabs}
\usepackage{array}  
\usepackage{rotating}
\usepackage{pifont} 
\usepackage{subcaption}
\usepackage{setspace}
\usepackage{hyperref}
\usepackage{natbib}  % 引用 BibTeX 需要 natbib 包
\usepackage{multirow} 
\usepackage{titlesec}
\usepackage{array} % 控制列宽
\usepackage{adjustbox} 
\usepackage{ragged2e} % 支持两端对齐
\bibliographystyle{chicago}  % 使用 Chicago (Author-Date) 引用格式

\begin{document}

\begin{titlepage}
    \centering
    \vspace*{2cm}
    {\Huge \textbf{Proposal: GIS-Powered Natural Language Query System}} \\
    \vspace{1.5cm}
    \textbf{Course:} DSCI 551 - Foundations of Data Science \\
    \textbf{Instructor:} Dr. Wensheng Wu \\
    \textbf{Institution:} Viterbi School of Engineering, University of Southern California \\
    \textbf{Date:} \today \\
    \vfill
    \textbf{Team Members:} \\
    - [Yucheng Liu] - [Project Designing, Data Collection, Document Writing] \\
    \vfill
    \begin{justify}
        \textbf{Project Abstract} - This proposal presents a \textbf{GIS-powered natural language query system} that integrates \textbf{PostGIS} and \textbf{LLMs} to allow users to retrieve geospatial data using natural language queries. The system translates user queries into \textbf{optimized SQL statements} that efficiently interact with a GIS database, enabling users to ask spatial questions such as **"Where are the nearest electric vehicle charging stations?"** or **"Find all Chinese restaurants within 5 km of my location."**. The project leverages \textbf{LLM-based query processing, spatial indexing, and GIS visualization tools} to enhance accessibility and usability of geospatial data.
        \vfill
        \textbf{Keywords:} GIS, LLM, PostGIS, Natural Language Queries, Spatial Database, Geospatial Search, Route Optimization, OpenStreetMap, API Development, Spatial Indexing, Machine Learning, Location-Based Services
    \end{justify}

\end{titlepage}

\section{Introduction}

\section{Data Source}

\section{Implementation}
\subsection{System Overview}
% 总体架构,描述 LLM + GIS 的流程

\subsection{LLM for Query Processing}
% LLM 如何解析用户查询并转换 SQL

\subsection{Database \& Storage}
% PostGIS 数据结构、GIS 查询优化

\subsection{API \& Integration}
% Flask API 如何提供服务

\subsection{Deployment Plan}
% 本地测试 vs. AWS 运行方式

\section{Team Members \& Roles}

\section{Timeline}

% 参考文献部分
\bibliography{references}  % 引用 BibTeX 文件 "references.bib"

\end{document}